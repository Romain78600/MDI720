\vspace{5mm}

{\fontsize{12pt}{22pt} \textbf{3. Moindres carrés:}\par}

\vspace{5mm}

10)  On suppose que X est de rang plein et on note $\hat{\theta}$ l’estimateur OLS. On note $\tilde{X}=(X_1,\hdots,X_p)$.On change l’échelle d’une des variables: $X_k$ est remplacé par $X_k b$, où $b>0$. \\
a) Soit $X_b=(1, X_1,\hdots, X_k b, \hdots, X_p)$. Montrer que $X_b = X D$ où $D$ est une matrice diagonale que l’on précisera. \\
On utilise simplement la définition de $X_b$ pour obtenir: \\
$X_b 
= \left( \begin{matrix} 1 & X_1 & \hdots & X_k b & \hdots & X_p \end{matrix} \right)
= \left( \begin{matrix} 1 \times 1 & 1 \times X_1 & \hdots & b \times X_k & \hdots & 1 \times X_p \end{matrix} \right)$ \\
$ \left( \begin{matrix} 1 & X_1 & \hdots & X_k & \hdots & X_p \end{matrix} \right) \times
\left( \begin{matrix} 1 & & & & & \\ & 1 & & & & \\ & & \ddots & & & \\ & & & b & & \\ & & & & \ddots & \\ & & & & & 1 \end{matrix} \right)
= X D$ \\ \\
$D$ est donc la matrice identité de dimension $p+1$ dont l'entrée diagonale $k+1$ est remplacée par $b$. \\
b)  Soit $\hat{\theta}_{b,n}$ l’estimateur OLS associé à $X_b$. Exprimer $\hat{\theta}_{b,n}$ en fonction de $\hat{\theta}_{n}$ et de $D$. \\
Par les équations normales, l'estimateur OLS $\hat{\theta}_{b,n}$ est égal à: \\
$\hat{\theta}_{b,n}
=(X_b^T X_b)^{-1}(X_b^T y)
=[(X D)^T (X D)]^{-1}[(X D)^T y]
=[D^T X^T X D]^{-1}[D^T X^T y]$ \\
$=[D^{-1} (X^T X)^{-1} (D^T)^{-1}] \ [D^T X^T y]
=D^{-1} (X^T X)^{-1} X^T y
=D^{-1} \hat{\theta}_{n}$ \\
Autrement dit, l'estimateur $\hat{\theta}_{b,n}$ est égal à l'estimateur $\hat{\theta}_{n}$ dont le coefficient $k+1$ a été multiplié par $1/b$. \\
c) Donner la variance de $\hat{\theta}_{b,n}$. \\
On utilise simplement les propriétés de la variance (et le fait que $D$ est diagonale) pour obtenir: \\
$Var(\hat{\theta}_{b,n})=Var(D^{-1} \hat{\theta}_{n})=(D^{-1})^2 Var(\hat{\theta}_{n})=\sigma^2 (D^{-1})^2 (X^T X)^{-1}$ \\
d) La prédiction donnée par le modèle est: \\
$\hat{y}_b
=X_b \hat{\theta}_{b,n}
=(X D) (D^{-1} \hat{\theta}_{n})
=X \hat{\theta}_{n}
=\hat{y}$ \\
Autrement dit, la prédiction n'est elle pas affectée par le changement d'échelle d'une des variables. \vspace{2mm}

11) Donner une formule explicite du problème $argmin_\theta \ \frac{1}{2} (y-X \theta)^T \Omega (y-X \theta)$ pour une matrice $\Omega=diag(w_1, \hdots, w_n)$ définie positive, dans le cas où $X$ est de plein rang. \\
On commence par développer la forme quadratique: \\
$\frac{1}{2} (y-X \theta)^T \Omega (y-X \theta)
=\frac{1}{2} \left[ y^T \Omega y + \theta^T X^T \Omega X \theta - 2 \theta^T X^T \Omega y \right]$ \\
(où pour obtenir le terme $2 \theta^T X^T \Omega y$ on a utilisé le fait qu'un scalaire est égal à sa transposée, et que $\Omega$ est symétrique). \\
Pour trouver l'argmin, on utilisera les règles suivantes de dérivées matricielles:\\
- Si $a$ et $b$ sont des vecteurs, on a: $\frac{\partial b^T a}{\partial b} = a$. Cela implique que: $\frac{\partial 2 \theta^T X^T \Omega y}{\partial \theta} = 2 X^T \Omega y$. \\
- Si $A$ est une matrice symétrique et $b$ un vecteur, on a: $\frac{\partial b^T A b}{\partial b} = 2 A b$. Cela implique que: $\frac{\partial \theta^T X^T \Omega X \theta}{\partial \theta} = 2 X^T \Omega X \theta$. \\
On conclut que: 
\begin{lflalign}
& \ \frac{\partial }{\partial \theta}\left( \ \frac{1}{2} (y-X \theta)^T \Omega (y-X \theta) \right)=0 \nonumber \\
\Leftrightarrow \ & \ \frac{\partial }{\partial \theta} \left( \frac{1}{2} \left[ y^T \Omega y + \theta^T X^T \Omega X \theta - 2 \theta^T X^T \Omega y \right] \right)=0 \nonumber \\
\Leftrightarrow \ & \ \frac{1}{2} \left( 2 X^T \Omega y - 2 X^T \Omega X \theta \right) = 0 \nonumber \\
\Leftrightarrow \ & \ X^T \Omega y - X^T \Omega X \theta = 0 \nonumber \\
\Leftrightarrow \ & \ X^T \Omega X \theta = X^T \Omega y \nonumber \\
\Leftrightarrow \ & \  \hat{\theta} = (X^T \Omega X)^{-1} (X^T \Omega y) \nonumber
\end{lflalign}


\vspace{5mm}

14) Dans le cas du modèle de régression avec design déterministe où X est de plein rang p, donner la valeur du risque de prédiction. \\

$$ Rpred(\hat{\theta}_n) = E\left[\| Y^* - \hat{Y} \|^2_2\right] $$
$$ Rpred(\hat{\theta}_n)  = E\left[\| X(\hat{\theta}_n - \theta^*) \|^2_2\right] $$
$$ Rpred(\hat{\theta}_n) = E \left[ \| X(X^TX)^{-1}X^T \epsilon \|^2_2 \right] $$
On pose $H_x = X(X^TX)^{-1}X^T$, on remarque que $H_x$ est un projecteur orthogonal et on écrit :
$$ Rpred(\hat{\theta}_n) = E\left[ \epsilon^TH_x\epsilon \right]$$
$$ = E\left[ tr(H_x \epsilon \epsilon^T)\right]$$
$$ = tr(H_x E(\epsilon\epsilon^T) $$ Comme $Cov(\epsilon) = \sigma^2 I_n$
$$ = \sigma^2 tr(H_x)$$

Comme $H_x$ est un projecteur orthogonal on a :
\smallbreak
$$
\begin{cases}
    H_x^T = H_x  \\
    H_x^2 = H_x
\end{cases} $$

Donc les valeurs propres de $\lambda_i$ de $H_x$ on pour propriété : 
$$\lambda_i^2 = \lambda_i \Longleftrightarrow 
\begin{cases}
    \lambda_i = 0  \\
    \lambda_i = 1
\end{cases} $$

Ainsi : 
$$ Rpred = \sigma^2 tr(H_x)$$
$$ = \sigma^2 \sum^n_{k=1} \lambda_k$$ avec $\lambda_k$ les vp de $H_x$
$$ = \sigma^2 rang(H_x)$$

Avec l'hypothèse de rang plein $Ker(X) = \{0\}$ on a :
$$dim(Vect(X)) = p$$
$H_x$ étant le projecteur orthgonal sur Vect(X), on obtient $rang(H_x) = p$
donc 
$$Rpred(\hat{\theta}_n) = \sigma^2 p$$



