\vspace{5mm}

{\fontsize{12pt}{22pt} \textbf{7. Test:}\par}

\vspace{5mm}

1) Pour des X1,...,Xn identiquement distribuées à valeur dans {0,1} décrire une procédure de test de l’hypothèse $p = P(X_1=1)=1/2$ contre son contraire.

On pose:
$$\begin{cases}
    H_0: p =P(X_1) = \frac{1}{2}  \\
    H_1: p = P(X_1=1) \neq \frac{1}{2}
\end{cases} $$
\\

\noindent
On choisit comme statistique de test

$$ T_i = \sqrt{n}\frac{\hat{p}- p}{\hat{\sigma}} $$
avec l'etismateur de l'espérance de X $$\hat{p} = \frac{1}{n}\sum^n_{i=1} X_i $$ , $p = \frac{1}{2}$

et $$\sigma^2 = \frac{1}{n}\sum^n_{i=1}{(X_i- \hat{p})^2} = \hat{p} - \hat{p}^2$$ 

\noindent
On suppose n assez grand pour que $$T_i \sim \mathcal{N}(0,\,1)\,.$$

\noindent
On note $\alpha$ notre niveau de précision. Pour ne pas rejeter l'hypothèse $H_0$, on doit avoir:
$$ \sqrt{n}\frac{\hat{p}-p}{\hat{\sigma}} <= t_{1-\frac{\alpha}{2}}$$ avec $t_{1-\frac{\alpha}{2}}$ le quantile de la loi normale centrée réduite

\noindent
On a la région de rejet R:
$$ R=[-t_{1-\frac{\alpha}{2}}; t_{1-\frac{\alpha}{2}}]$$
