\vspace{5mm}

{\fontsize{12pt}{22pt} \textbf{4. Ridge:}\par}

\vspace{5mm}

On note $\hat \theta = {\rm arg} \displaystyle\min_{\theta} \frac{1}{2}\| \mathbf{y} -X \theta \|_2^2+ \frac{\lambda}{2}\|\theta\|_2^2$ l’estimateur Ridge

Soit $f: \theta \mapsto \frac{1}{2}||Y-X\theta||_2^2+\frac{\lambda}{2}||\theta||_2^2$ \\

1) Quand $X=Id_n$ on a $n=p+1$ et $f(\theta)=\frac{1}{2}||Y-Id_n\theta||_2^2+\frac{\lambda}{2}||\theta||_2^2$ \\

Pour des questions de notations, on utilisera $Id_n=Id_{p+1}=Id$ \\

$f(\theta)=\frac{1}{2}\Sigma_{i=1}^n(Y_i-Id_i\theta)^2+\frac{\lambda}{2} \Sigma_{i=1}^n \theta_i^2$ \\

Le minimum de la fonction est atteint lorsque $\nabla f(\theta) = 0$ \\

$\nabla f(\theta) = 0 => \forall k=1,...,p: \frac{\partial f(\theta)}{\partial \theta_k}=0$ \\

$\frac{\partial f(\theta)}{\partial \theta_k}=2\frac{1}{2} (-Id_{i,k}) \Sigma_{i=1}^n (Y_i-\Sigma_{j=1}^pId_{i,j}\theta_j)+2\frac{\lambda}{2} \theta_k$ \\

$~~~~~~~=-\Sigma_{i=1}^n(Id_{i,k}Y_i-Id_{i,k}Id_i\theta)+\lambda \theta_k$ \\

Donc $\nabla f(\theta)=-\Sigma_{i=1}^n(Id_i^TY_i-Id_i^TId_i\theta)+\lambda \theta=-\Sigma_{i=1}^nId_i^TY_i+\Sigma_{i=1}^nId_i^TId_i \theta +\lambda \theta$ \\

$~~~~~~~~~~~~~~~~~=-Y+\theta+\lambda \theta$ \\

$\nabla f(\theta) = 0 => \theta = \frac{1}{1+\lambda}Y= \hat{\theta}_{n,\lambda}^{Ridge} $\\


2) Pour $X \in \mathbb{R}^{n \times p}$ quelconque, $f(\theta)=\frac{1}{2}\Sigma_{i=1}^n(Y_i-X_i\theta)^2+\frac{\lambda}{2} \Sigma_{i=1}^n \theta_i^2$ \\

Comme pour la question 1), \\
$\nabla f(\theta) = 0 => \forall k=1,...,p: \frac{\partial f(\theta)}{\partial \theta_k}=0$ \\

$\frac{\partial f(\theta)}{\partial \theta_k}=2\frac{1}{2} (-X_{i,k}) \Sigma_{i=1}^n (Y_i-\Sigma_{j=1}^pX_{i,j}\theta_j)+2\frac{\lambda}{2} \theta_k$ \\

$~~~~~~~=-\Sigma_{i=1}^n(X_{i,k}Y_i-X_{i,k}X_i\theta)+\lambda \theta_k$ \\

Donc $\nabla f(\theta)=-\Sigma_{i=1}^n(X_i^TY_i-X_i^TX_i\theta)+\lambda \theta=-X^TY+X^TX\theta + \lambda \theta$ \\

Ainsi $\nabla f(\theta)=0 => -X^TY+X^TX\theta + \lambda \theta=0$ \\
$=> (X^TX + \lambda Id_n)\theta = X^TY => \theta = (X^TX + \lambda Id_n)^{-1} X^TY = \hat{\theta}_{n,\lambda}^{Ridge} $ \\

3) Donner la variance de l'estimateur Ridge sous l’hypothèse que le bruit $\mathbf{y} -X \theta^*$ est centré et de variance $\sigma^2 {\rm Id}_n$.

$$\hat \theta_{\lambda}^{rdg} = (X^\top X+\lambda Id_p)^{-1}X^\top Y$$
$$Var(\hat \theta_{\lambda}^{rdg}) = Var((X^\top X+\lambda Id_p)^{-1}X^\top Y)$$ 
$$ = ((X^\top X+\lambda Id_p)^{-1}X^\top) Var(Y) ((X^\top X+\lambda Id_p)^{-1}X^\top)^\top$$ 
$$ = (X^\top X+\lambda Id_p)^{-1}X^\top Var(X \theta^\star + \epsilon)X(X^\top X+\lambda Id_p)^{-1}$$ 
$$ = (X^\top X+\lambda Id_p)^{-1}X^\top Var(\epsilon)X(X^\top X+\lambda Id_p)^{-1}$$ 
$$ = \sigma^2 (X^\top X+\lambda Id_p)^{-1}X^\top X(X^\top X+\lambda Id_p)^{-1}$$ 

En remarquant que $X^\top X$ et $(X^\top X+\lambda Id_p)^{-1}$ sont diagonalisables dans la même base, on pose:

$$(X^\top X+\lambda Id_p)^{-1} = PA^{-1}P^{-1}$$
$$X^\top X = PBP^{-1}$$

Avec $A^{-1}$ et $B$ deux matrices diagonales.

On a alors:

$$Var(\hat \theta_{\lambda}^{rdg}) = \sigma^2PA^{-1}P^{-1}PBP^{-1}PA^{-1}P^{-1}$$
$$ = \sigma^2PA^{-1}BA^{-1}P^{-1}$$
$$ = \sigma^2PA^{-1}A^{-1}BP^{-1}$$
$$ = \sigma^2PA^{-1}Id_pA^{-1}Id_pBP^{-1}$$
$$ = \sigma^2PA^{-1}P^{-1}PA^{-1}P^{-1}PBP^{-1}$$
$$ = \sigma^2 (X^\top X+\lambda Id_p)^{-1} (X^\top X+\lambda Id_p)^{-1}X^\top X$$
$$ = \sigma^2 (X^\top X+\lambda Id_p)^{-2} X^\top X$$






